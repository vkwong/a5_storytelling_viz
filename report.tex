\documentclass[12pt,fullpage]{article} 
\usepackage{graphicx,latexsym} 
\usepackage{longtable} 
\usepackage{hyperref} 
\usepackage{url}
\usepackage{fullpage, setspace} %an example of how to specify two packages at once.
\graphicspath{ {/} }
\hypersetup{
    colorlinks=true,
    linkcolor=blue,
    urlcolor=blue,
} 
\title{Assignment 5 Storytelling Visualization: Report} 
\author{Vivian Wong} 
\begin{document} 
\maketitle %this command makes a title page using the \title and \author information in the preamble. 
URL: \url{https://vkwong.github.io}\\
 
I decided to make a visualization of how California counties voted in the 2016 election for Erica Joy's Medium article \href{https://medium.com/@ericajoy/after-87ce712b3645}{After}. When Erica is talking about the disbelief that everyone had that Trump won especially in Silicon Valley, she had quantitative evidence to back up the fact that actually a lot more people voted for Trump in Silicon Valley than you would think. I chose to put my visualization right after to emphasize her evidence.\\
 
The visualization focuses on how many Trump voters there are in each county. I took the numbers from Politico's \href{https://www.politico.com/2016-election/results/map/president/california/}{2016 California Presidential Election Results}. When you hover over a county, it is outlined and a tooltip appears listing the county name, the number of trump votes in the county, and the percentage of trump voters. I listed the count first because it always feels shocking to read such a large count. I also omitted displaying any Clinton statistics to magnify how big the numbers seem.\\
 
For the color scheme, it might seem that California as a whole is colored extremely red. However, if you take into account the article?s tone and tolerance of Trump, it is colored correctly.
\end{document}